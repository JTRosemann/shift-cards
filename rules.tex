\documentclass[a4paper]{article}

\usepackage[utf8]{inputenc}
\usepackage[T1]{fontenc}
\usepackage[ngerman]{babel}
\pagestyle{empty}

\usepackage{xstring}
\usepackage{xcolor}
\usepackage{booktabs}
\usepackage{array}
\usepackage{geometry}


\IfStrEqCase{ninja}{
  {cool}{
    \definecolor{bone}{HTML}{104210}
    \definecolor{btwo}{HTML}{7A871E}
    \definecolor{mone}{HTML}{E55B13}
    \definecolor{mtwo}{HTML}{F6A21E}
  }
  {christmas}{%https://www.color-hex.com/color-palette/1006651
    \definecolor{bone}{HTML}{274e13}
    \definecolor{btwo}{HTML}{08a217}
    \definecolor{mone}{HTML}{990000}
    \definecolor{mtwo}{HTML}{cc0000}
    \definecolor{othe}{HTML}{eeeeee}
  }
  {ninja}{%https://www.color-hex.com/color-palette/1006625
    \definecolor{bone}{HTML}{3d85c6}
    \definecolor{btwo}{HTML}{6a329f}
    \definecolor{mone}{HTML}{cc0000}
    \definecolor{mtwo}{HTML}{ce7e00}
    \definecolor{othe}{HTML}{38761e}
  }
  {summer}{%https://www.color-hex.com/color-palette/1006598
    \definecolor{bone}{HTML}{1b998b}
    \definecolor{btwo}{HTML}{c5d86b}
    \definecolor{mone}{HTML}{d7262d}
    \definecolor{mtwo}{HTML}{f46036}
    \definecolor{othe}{HTML}{2e294e}
  }
  {fox}{%https://www.color-hex.com/color-palette/1006762
    \definecolor{bone}{HTML}{274e13}
    \definecolor{btwo}{HTML}{066052}
    \definecolor{mone}{HTML}{990000}
    \definecolor{mtwo}{HTML}{ef2828}
    \definecolor{othe}{HTML}{b6d7a8}
  }
  {oui}{%https://www.color-hex.com/color-palette/1006585
    \definecolor{bone}{HTML}{164b70}
    \definecolor{btwo}{HTML}{12598b}
    \definecolor{mone}{HTML}{851414}
    \definecolor{mtwo}{HTML}{a84a4a}
    \definecolor{othe}{HTML}{d5e2eb}
  }
}

\newcommand{\bonusA}{\textcolor{bone}{\textbf{Bonus~A}}}
\newcommand{\bonusB}{\textcolor{btwo}{\textbf{Bonus~B}}}
\newcommand{\malusA}{\textcolor{mone}{\textbf{Malus~A}}}
\newcommand{\malusB}{\textcolor{mtwo}{\textbf{Malus~B}}}

\newcommand{\othe}[1]{\textcolor{othe}{\textbf{#1}}}

\newcommand{\kapital}{\textit{\othe{Kapital}}}
\newcommand{\status}{\othe{Status}}


\title{Umverteilung -- die Erweiterung für jedes Spiel}
\date{}
\begin{document}
\maketitle
\large
So funktioniert es:
Als erstes wird gemeinsam die umseitige Tabelle ausgefülllt.
In das Feld \othe{Trigger} wird eingetragen, wann eine Umverteilungs-Karte gespielt wird.
Das kann zum Beispiel alle paar Runden passieren oder wenn im Spiel eine bestimmte Situation auftritt.
\bonusA{}, \malusA{}, etc. geben die Effekte an, die einmalig ausgeführt werden, wenn sie auf einer ausgespielten Karte beschrieben sind.
Bei \status{} wird das Erfolgsmaß definiert, welches entscheidet wer welchen Effekt erfährt.
Manche Karten verlangen, dass der \status{} eine Totalordnung induziert.
Ist dies nicht der Fall, müssen diese Karten aus dem Spiel genommen werden.
Manche Karten verlangen zusätzlich auch eine Definition für \kapital{} oder \kapital{}-Verteilung.
Diese kann nun auch angegeben werden.
Das können zum Beispiel Handkarten oder Spielgeld sein.

Nun werden die Karten ausgesucht mit denen gespielt werden soll.
Dabei ist darauf zu achten, dass die Vorraussetzungen unten in dem gelben Feld erfüllt sind.

Die Karten werden gemischt und als verdeckter Stapel auf den Tisch gelegt.
Immer wenn der Trigger ausgelöst wird eine Umverteilungs-Karte vom Stapel ausgespielt und der Effekt ausgeführt.

Das Spiel kann jederzeit um weitere Karten erweitert werden.

\end{document}
