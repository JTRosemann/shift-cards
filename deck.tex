
\newcommand{\cardsoc}[3]{
  \fullcard{#1}{red}{Gleichmacherei}{#2}{#3}
}

\newcommand{\cardlib}[3]{
  \fullcard{#1}{black}{Ungleichmacherei}{#2}{#3}
}

\newcommand{\cardMedianBonMal}[3]{
  \cardlib{#1}{Alle echt oberhalb des \status{}-Medians bekommen {#2} und alle echt unterhalb dessen bekommen {#3}.}{total geordneter \status{}}
}

\newcommand{\cardMedianMalBon}[3]{
  \cardsoc{#1}{Alle echt oberhalb des \status{}-Medians bekommen {#2} und alle echt unterhalb dessen bekommen {#3}.}{total geordneter \status{}}
}

%%%%%%%%%%%%%%%%%%%%%%%%%
%% THIS is the DECK
%%%%%%%%%%%%%%%%%%%%%%%%%
\newcommand{\cardA}{\cardMedianBonMal{Horse-Sparrow-E. A}{\bonusA{}}{\malusA}}

\newcommand{\cardB}{\cardMedianBonMal{Horse-Sparrow-E. B}{\bonusB{}}{\malusB}}

\newcommand{\cardC}{\cardMedianMalBon{Umverteilung A}{\malusA{}}{\bonusA{}}}

\newcommand{\cardD}{\cardMedianMalBon{Umverteilung B}{\malusB{}}{\bonusB{}}}

\newcommand{\cardE}{
  \cardlib{Lobbyistendinner}{Alle echt oberhalb des \status{}-Medians dürfen sich auf eine Karte einigen, die ausgespielt wird.}{total geordneter \status{}}
}

\newcommand{\cardF}{
  \cardsoc{Graswurzelbew.}{Alle echt unterhalb des \status{}-Medians dürfen sich auf eine Karte einigen. Wenn sich auch die Mehrheit derer echt oberhalb dessen sich dafür ausspricht, wird sie ausgespielt.}{total geordneter \status{}}
}

\newcommand{\cardG}{
  \cardsoc{Kommunismus}{Alles \kapital{} wird eingesammelt und gleichmäßig wieder aufgeteilt.}{faire \kapital{}-Veteilung}
}

\newcommand{\cardH}{
  \cardsoc{Staatseigentum}{Alles \kapital{} wird eingesammelt und offen in die Mitte gelegt. Jede Nutzung wird muss im Plenum abgestimmt werden. Diese Karte bleibt liegen bis \textit{Privateigentum} gespielt wird.}{\kapital{}}
}

\newcommand{\cardI}{
  \cardlib{Privateigentum}{Löst \textit{Staatseigentum} auf. Das gemeinschaftliche \kapital{} wird fair wieder aufgeteilt.}{faire \kapital{} Verteilung}
}

\newcommand{\cardJ}{
  \cardsoc{Revolution}{Jede Person tauscht mit der ihr in der \status{}-Rangliste diametral gegenüber stehenden Person den Platz und den gesamten Spielzustand (also auch Spielfigur, Karten, Punktestand,~...)}{total geordneter \status{}}
}

\newcommand{\cardK}{
  \cardsoc{Vollversammlung}{Gemeinsam wird eine Person beschlossen, die \malusA{} bekommnt und eine die \bonusA{} bekommt.}{$\geq$ 3 Spieler}
}

\newcommand{\cardL}{
  \cardlib{Oligarchie}{Alle legen alles Kapital in die Mitte. Danach wird das Kapital fair unter allen echt oberhalb des \status{}-Medians wieder aufgeteilt. (Wenn niemand echt oberhalb des Medians ist, wird es unter allen fair aufgeteilt.}{\footnotesize total geordneter \status{},\\\footnotesize faire \kapital{} Verteilung}
}
