
\newcommand{\cardA}{
  \cardsoc{Chacun}{À chacun son tour, \bonusA{} un joueur place une  cartes de son choix de sa main à ce qu'il pense être sa bonne position dans la frise.\\
    Une fois posée, la carte est retournée : si \malusA{} elle est en bonne position, la carte est laissée face visible pour compléter la frise, sinon, la carte est défaussée et le joueur en tire une nouvelle.\\
    Le premier joueur a avoir placé correctement toutes ses cartes est déclaré vainqueur. Si la pioche est épuisée alors qu'un joueur doit tirer à nouveau, il est déclaré perdant.}
}

\newcommand{\cardB}{
  \cardlib{Chacun}{À chacun son tour, un joueur place une  cartes de son choix de sa main à ce qu'il pense être sa bonne position dans la frise.\\
    Une fois posée, la carte est retournée : si elle est en bonne position, la carte est laissée face visible pour compléter la frise, sinon, la carte est défaussée et le joueur en tire une nouvelle.\\
    Le premier joueur a avoir placé correctement toutes ses cartes est déclaré vainqueur. Si la pioche est épuisée alors qu'un joueur doit tirer à nouveau, il est déclaré perdant.}
}
